\noindent The previous section highlighted the main outcomes of the developed model. The analysis enabled to appreciate how the five indicators evolved over time under the municipality's intervention. In this section, the model's limitations are acknowledged and discussed in order to identify opportunities for improvements. \\

\noindent First, the model analysis showed that the digital campaign is very effective for all different demographic groups, except for elderly. Elderly are rarely tech-savvy, but they are more sensitive to physical activities like billboards. During a year when budget is dedicated to digital campaign, all groups increase their perception, importance and knowledge score in favour of recycling, except elderly people. The next year, a billboard campaign is launched since elderly household's recycling score highly differs from the other groups. The reliability of a model relying on only two types of measures to reach recycling target can be questioned. Only using 2 types of campaigns would lead to a very monotonic campaign policy. Instead variety of innovative policies should be stimulated to improve the relevance of such model. In the same vein, some could argue that the desired effect of the measures could fade away due to their redundancy. As people get used to the same approach being repeated, their attention and responsiveness would certainly decrease. \\

\noindent Second, when a certain plastic recycling rate increase due to the activities of the municipality, the rate remains stable even if no activity is performed. In a realistic world the rate would decrease when there is less or even no effort from the municipality to promote behaviours in favour of plastic recycling. Therefore, the recycling rate of the municipality should go down when there is a period without activity. //

\noindent Third, the current model launches only one campaign possible per year. In reality, several activities could be combined to trigger synergies. For instance, a communication campaign associated with a tax program and infrastructure plan would have a stronger impact on the recycling rate. \\

\noindent Finally, the last limitations discussed here concern the effect caused by the measures undertaken by the municipalities. All activities in the model only have a positive impact on the recycling rate. In reality, policies could be unpopular and generate resistance among the population. \\

