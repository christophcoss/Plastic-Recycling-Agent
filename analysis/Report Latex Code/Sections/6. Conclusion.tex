To increase society's chances to fully transition from a linear to a circular economy, every actor in society must embrace the concepts of this new paradigm that designs out waste from socio-economic systems. In the Netherlands, municipalities play a crucial role in steering this transition for citizens. Therefore, this project aims to investigate the trade-offs in the policy space to achieve the highest possible plastic recycling rate. Agent Based Modeling is the approach used for this research project to address the aforementioned challenge. Thus, a full modelling cycle was applied in order to build a model that simulates the behaviour of agents involved in the municipal waste recycling activities. The model was defined, developed, tested and eventually validated. Then, model's outcomes were analysed to derive relevant advises to support the local government of the municipality of Delft to undertake promising actions that could boost the recycling rate of plastic. The model shows insightful results regarding the evolution of the recycling rate in respect with waste and plastic collection. In addition, the measures modelled proved to have a clear impact on plastic recycling. However, the report identified several limitations that are opportunities to further improve the model's usefulness in guiding civil servants in their mission of phasing out waste from households. 