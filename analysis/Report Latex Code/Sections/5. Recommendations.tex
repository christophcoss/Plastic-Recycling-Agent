\noindent In this chapter the recommendations following from the model will be elaborated. 

\subsection{Recommendations on refining the model}
\noindent As described in the the discussion there are multiple flaws which could be implemented in a more realistic version of the model. 
\noindent First the knowledge, perception and importance of the households should go down when there is no activity performed by the municipality. This could be implemented by introducing a function to the knowledge, perception and importance scores of the households, which results in a decrease when there is no activity.
\noindent Secondly the effect of similar activities should decrease over time, and be reset when the activity has not been used for one year. This could be implemented by introducing a if statement to check if the former activity was similar to the new activity. If that is the case, the effect of the activity can be reduced by a factor. 
\noindent Thirdly the municipality should be able to buy multiple activities. This can be done by making multiple activities in which the current activities are combined. This would lead to a variety of combinations. However, it should also calculate that when there is an overkill of activities combined for a longer period of time that the effect will decrease. 

\subsection{Recommendations on policy}
\noindent Since the model has quite some logical flaws, it would be quite hard to give some actual policy advise on what policy to use. To make this model a model which can be used to derive some actual recommendations with, the next things are needed.
\noindent The first thing is to update the model with the above described recommendations on refining the model. 
\noindent The second advise for the municipality would be to set up an extensive research on the effect of different campaigns. The thing to keep in mind here, is that it is extremely hard to quantify the effect of hypothetical campaigns on the households. Several ways of doing this, is by performing literature research, putting out surveys under the citizens, and look to empirical data where data from past campaigns can be analysed. 
\noindent Finally it is also very important to make a very accurate zero measurement on how well the different demographic groups recycle in the municipality. This can be done by interviewing or surveys. By doing this, combined with an accurate as possible quantification of the activities, the model can be as accurate as possible. 
\noindent Final thing to mention, even if the model gets the upgrades mentioned in the recommendation section, it still is very hard to give a certain picture of the future. There are a lot of unforeseen things that could happen that disrupts the system, causing the model to be wrong. 