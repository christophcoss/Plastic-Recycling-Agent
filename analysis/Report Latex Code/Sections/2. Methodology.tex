\noindent As presented in the introduction, Agent-Based Modelling is the methodology used to model the complex adaptive system of waste recycling in the Dutch municipality of Delft. An agent-based model is a computational representation of the real world. Such model consists of individual 'agents' interacting according to a set of rules, wherefore the effect (i.e., behaviours) of these rules can be examined. For the purpose of this research project, an agent-based model is well equipped to simulate and understand the potential effects of policy changes on society and plastic recycling rate. With this approach, the authors aim to discover the trade-offs and synergies related to policy measures and try to deliver recommendations to policy makers accordingly.\\

\noindent To deliver a comprehensive agent-based model, the modelling cycle has been applied as follow. First, the conceptual model is defined. This stage starts with the definition of the model’s goal and the research questions derived from this goal. The approaches to address the research questions are specified in parallel. The conceptual phase also includes the definition of process, concepts, features, agents, agent-environment interactions, and the visualisation of the model. Second, the conceptual model is translated into a formal model to explain how the research case is modelled. This phase takes place right before implementing the code. Working on a formal model requires to develop variables and functions from the concepts and relationships defined in the conceptual model. Once the formal model is complete, it can be implemented into a code. To write the code, this project uses the Mesa framework available with Python programming language. Third, the modelling cycle covers the experimental runs of the model. These experiments aim to verify and validate the behaviour of the model before interpreting the model’s output. Finally, the modelling cycle ends with simulations and data analysis. During this last stage, insights are derived from the model output.\\

\noindent In the next section “Modelling cycle”, the report will walk you through each step of the cycle to explain the development of the model from the research objectives to the output analysis.\\

